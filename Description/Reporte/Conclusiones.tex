\newpage
%%%%%%%%%%%%%%%%%%%
\section{Conclusiones}
%%%%%%%%%%%%%%%%%%%

\begin{itemize}

\item Una de las pricipales conclusiones que se obtienen al realizar el presente proyecto es comprender la importancia del uso de segmentaci\' on en los procesadores actuales en tanto que se puede obtener un desempe\~ no mucho mayor por el simple hecho de utilizar cada uno de los segmentos del procesador para tareas independientes y que de esta manera no se desperdicie tiempo de procesamiento, como si ocurre en los procesadores de ciclo simple.
\item Adem\' as de comprender el funcionamiento de un procesador segmentado mediante su implementaci\' on y dise\~ no en verilog, tambi\' en se logr\' o entender y comprender el efecto de los diferentes \textit{hazards} que se presentan aqu\' i y complican la labor de su desarrollo. De esta manera, el uso de unidades de \textit{forwarding} para procesadores segmentados es un deber que deber\' ia de tener siempre y cuando se busqu\' e mejorar el desempe\~ no del mismo.
\item Al igual que los puntos anteriores, el uso de m\' as de un registro en el procesador hace que el funcionamiento sea considerablemente m\' as complicado pues es necesario tomar en cuenta dependencia de instrucciones que normalmente no producen \textit{hazards}, tanto por el operando 1 como el operando 2.
\item Para lograr implementar una unidad l\' ogico arim\' etica, es necesario contar con un controlador de la misma, por lo cual una conclusi\' on que se obtuvo al respecto es que debe de existir un diagrama arquitect\' onico y un conjunto de instrucciones definidas mucho antes de realizar la implementaci\' on, y si se desean agregar nuevas instrucciones puede resultar en un cambio considerable del c\' odigo de programaci\' on.
\item Las instrucciones de ensamblador propuestas para el procesador son limitantes para diferentes aplicaciones y usos de las mismas, principalmente por no tener direccionamiento indirecto, interrupciones ni pila.
\item Por medio del uso de banderas se puede efectivamente determinar cuando un branch debe ser tomado.
\item Con respecto a la distribuci\' on de trabajo en grupo de logr\' o determinar que la distribuci\' on de las tareas debe ir de la mano con la modulaci\' on del esquema principal, lo cual simplifica mucho la ejecuci\' on del trabajo tanto aqu\' i como en cualquier otro proyecto pues permite a cada uno de los integrantes trabajar de manera independiente y tener un molde al cual apegarse para que el tiempo utilizado en la etapa de desarrollo sea lo menor posible.

\end{itemize}



