%%%%%%%%%%%%%%
\newpage
\section{Estrategia de pruebas}

\par Con el fin de comprobar el adecuado comportamiento del procesador con etapas segmentadas, se define el siguiente plan de pruebas el cual busca comprobrar que adem\' as de que cada m\' odulo funciona de manera adecuanda por separado, tambi\' en que todo el procesador funciona bien en cualquier caso, con solo evaluar los puntos cr\' iticos.

Se debe comprobar que cada m\' odulo dise\~ nado cumple con lo establecido en el diagrama arquitect\' onico y en la definici\' on dada en la secci\' on de Dise\~ no, donde los resultados de la simulaci\' on obteniddos deben mostar lo que se observa en la Tabla \ref{tab:estrategia} y \ref{tab:estrategia1}.

\begin{table}[h]
%\begin{longtable}[h]
\begin{tabularx}{\textwidth}{l X}
%\begin{tabulary}{1.0\textwidth}{ll}
{\bf M\' odulo}                           & {\bf Resultado Esperado}                            \\
{\it Contador de programa}                & Debe mostrar un diagrama temporal como el PC funciona de manera sincr\' onica de acuerdo a una se\~ nal de reloj del sistema. Adem\' as debbe mostrar que la salida corresponde a direcciones de memoria sin signo y de 10 bits en formato decimal.                   \\
{\it Memoria de instrucciones y de datos} & Se debe mostrar el acceso de al menos 10 posiciones de memoria diferentes, incluyendo la primer posici\' on y la \' ultima, y se debe mostrar el contenido de cada posici\' on. Adem\' as se debe intentar ingresar a una posici\' oon inv\' alida (negativa y mayor a 1024) y comprobar que no se puede acceder                           \\
{\it Decodificador de instrucciones}      & Se debe mostrar la decodificaci\' on  de todas las se\~ nales de control para todas las instrucciones posibles. Recuerde que todas las se\~ nales deben estar definidas y cualquier \textit{switch case} debe contener su respectivo \textit{default state}, esto para que el compilador infiera un RTL. Las se\~ nales de control a mostrar necesariamente son: \textit{write to a}, \textit{write to b}, \textit{Mux Pre ALU A}, \textit{Mux Pre ALU B}, \textit{branch taken}, \textit{jump}, \textit{write mem}, \textit{R/W} y \textit{write back mux}.              \\

\end{tabularx}
%\end{tabulary}
\caption{Estrategia de pruebas}
\label{tab:estrategia}
\end{table}


\begin{table}[h]
%\begin{longtable}[h]
\begin{tabularx}{\textwidth}{l X}
%\begin{tabulary}{1.0\textwidth}{ll}

{\it Registros A y B}                     & Se debe mostrar de manera independiente que se puede almacenar valores distintos y colocar su valor en la salida del m\' odulo dependiento de la combinaci\' on de las se\~ nales de control: \textit{write to a} y \textit{write to b}. Por lo tanto, se debe hacer una simulaci\' on con todas las posibles combinaciones de las se\~ nales de control.            \\
{\it M\' odulo de salto (branch)}         & Se debe mostrar el funcionamiento de este m\' odulo para todas las posibles intrucciones de salto, tomando en cuenta las combinaciones de banderas que hacen que el \textit{branch} se tome y no se tome, incluyendo las instrucciones JUMP y NOP. Es importante destacar que se debe mostrar que el PC nuevo es la suma de PC actual m\' as la magnitud del salto cuando es suma, o resta cuando es substracci\' oon y adem\' as se debe mostrar que PC nuevo es PC actual + 1 cuando el salto no se toma. As\' i, por cada instrucci\' on hay que simular 3 combinaciones posibles, tomando el salto y sumando direcci\' on, tomando el salto y restando direcci\' on, o no tomando el salto.                   \\


{\it M\' odulo Controlador de ALU}        & Se debe mostrar la decodificaci\' on para todas las instrucciones del procesador, su respectivo valor de entrada de control a la ALU y su respectiva salida en un diagrama temporal. Recuerde que es un m\' odulo completamente combinacional por lo que las se\~ nales no est\' an relacionadas con ning\' un reloj del sistema. \\
{\it M\' odulo ALU}                       & Se debe mostrar un un diagrama temporal todas las posibles operaciones que se puedan realizar en la ALU, as\' i como tambi\' en el caso en el que la ALU no desarrolla ninguna tarea. Dado que las entradas siempre son positivas, todos los 10 bits de entrada representan la magnitud de un valor, si el resultado de la substracci\' on es menor que cero, se debe indiciar de alguna manera,  que la bandera correspondiente debe ser ajustada.                                                    \\                                                
\end{tabularx}
%\end{tabulary}
\caption{Estrategia de pruebas, continuaci\' on}
\label{tab:estrategia1}
\end{table}
%\end{longtable}

