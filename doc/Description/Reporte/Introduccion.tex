%%%%%%%%%%%%%%%%%%%%%%
\section{Introducci\' on}
%%%%%%%%%%%%%%%%%%%%%%

El prop\' osito de este laboratorio consiste en brindar una metodolog\' ia de trabajo en la que se permita segmentar el proceso de dise\~ no e implementaci\' on de un proyecto de software en diferentes etapas, y adem\' as, delegar diferentes responsabilidades entre cada uno de los autores para simplificar el desarrollo de las subtareas.\\

Se busca dise\~ nar e implementar un procesador con pipeline de cinco etapas para dos registros de 16 bits, utilizando el lenguaje de descripci\' on de hardware Verilog. Las etapas definidas para el procesador son: Instruction Fetch (IF), Instruction Decode (ID), Execution (EX), Memory (MEM) y Write Back (WB).\\

Finalmente, se busca dar a conocer una posible opci\' on de organizaci\' on temporal para que el proyecto cumpla con especificaciones de fechas de entrega y diferentes hitos, para lo cual se dise\~ na un esquema propuesto y luego se contrasta con el desarrollo real del proyecto. \\

A modo de resumen, se logr\' o concluir con el laboratorio de manera satifasctoria mediante la implementaci\' on de diferentes m\' odulos en Verilog, y su respectivo plan de pruebas y cronograma de tareas. Adem\' as, se desarrollaron efectivamente habilidades en las \' areas de dise\~ no, modelado y verificaci\' on de hardware en Verilog de circuitos digitales.\\ 

\setcounter{secnumdepth}{4}
